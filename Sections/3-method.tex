\chapter{Proposed methods}
\label{ch:methods}

In this section I will describe the methods I am employing in my research with some sections intentionally brief.
This section is heavily supplemented by publications included as appendices to this document.

\section{Deterministic transport}

For deterministic methods, Appendix~\ref{app:therefore} contains expanded derivations of the higher order space-time-energy discretization scheme, derivation of Fourier analysis of the discretization scheme, derivation of the method of manufactured solution, and description of the implementation in code.
The remainder of this section will be devoted to describing the current state of investigations into an acceleration scheme for OCI in the thin limit (see section \ref{ssec:method_acc}).

\subsection{Derivation of Higher order Methods in Iterative Schemes}

When the S$_N$ transport equation is descritzed in space using simple corner balance and in time with time dependent multiple balance a 4-equation system is begot:
\begin{subequations}
    \label{eq:tdmb+scb}
    \begin{multline}
    \label{eq:scb-mb-a}
    \frac{\Delta x_j}{2} \frac{1}{v_g} \left( \frac{\psi_{m,g,k+1/2,j,L} - \psi_{m,g,k-1/2,j,L}}{\Delta t} \right) \\
     + \mu_m \left[ \frac{\left( \psi_{m,g,k,j,L} + \psi_{m,g,k,j,R} \right)}{2}  - \psi_{m,g,k,j-1/2} \right] \\
    + \frac{\Delta x_j}{2} \Sigma_{j} \psi_{m,g,k,j,L} 
    = \frac{\Delta x_j}{2} \frac{1}{2} \left( \sum\limits_{g' = 0}^G \Sigma_{s,j, g'\to g}(x) \sum\limits_{n=1}^N w_n \psi_{n,g,k,j,L} + Q_{k,j,L} \right) \;,
    \end{multline}  
    \begin{multline}
    \label{eq:scb-mb-b}
    \frac{\Delta x_j}{2} \frac{1}{v} \left( \frac{\psi_{m,k+1/2,j,R} - \psi_{m,k-1/2,j,R}}{\Delta t} \right) + \\
    \mu_m \left[ \psi_{m,k,j+1/2} - \frac{\left( \psi_{m,k,j,L} + \psi_{m,k,j,R} \right)}{2}   \right] \\
    + \frac{\Delta x_j}{2} \Sigma_{j} \psi_{m,k,j,R} = \frac{\Delta x_j}{2} \frac{1}{2} \left( \sum\limits_{g' = 0}^G \Sigma_{s,j, g'\to g} \sum\limits_{n=1}^N w_n \psi_{n,g',k,j,R} + Q_{k,j,R} \right) \;,
    \end{multline}  
    \begin{multline}
    \label{eq:scb-mb-c}
    \frac{\Delta x_j}{2} \frac{1}{v_g} \left( \frac{\psi_{m,g,k+1/2,j,L} - \psi_{m,g,k,j,L}}{\Delta t/2} \right) \\
    + \mu_m \left[ \frac{\left( \psi_{m,g,k+1/2,j,L} + \psi_{m,g,k+1/2,j,R} \right)}{2}  - \psi_{m,g,k+1/2,j-1/2} \right]\\
    + \frac{\Delta x_j}{2} \Sigma_{j} \psi_{m,g,k+1/2,j,L} = \frac{\Delta x_j}{2} \frac{1}{2} \left( \sum\limits_{g' = 0}^G \Sigma_{s,j, g'\to g} \sum\limits_{n=1}^N w_n \psi_{n,g',k+1/2,j,L} + Q_{k+1/2,j,L} \right) \;,
    \end{multline}    
    \begin{multline}
    \label{eq:scb-mb-d}
    \frac{\Delta x_j}{2} \frac{1}{v_g} \left( \frac{\psi_{m,g,k+1/2,j,R} - \psi_{m,g,k,j,R}}{\Delta t/2} \right) + \\
    \mu_m \left[ \psi_{m,g,k+1/2,j+1/2} - \frac{\left( \psi_{m,g,k+1/2,j,L} + \psi_{m,g,k+1/2,j,R} \right)}{2}   \right]  \\
    + \frac{\Delta x_j}{2} \Sigma_{j} \psi_{m,g,k+1/2,j,R} = \frac{\Delta x_j}{2} \frac{1}{2} \left( \sum\limits_{g' = 0}^G \Sigma_{s,j, g'\to g} \sum\limits_{n=1}^N w_n \psi_{n,g',k+1/2,j,R} + Q_{k+1/2,j,g,R} \right) \;,
    \end{multline} 
\end{subequations}
%
where $\Delta x$ is the cell width, $j$ is the spatial index, $L$ and $R$ denote the right and left half-cell averaged quantities, $k$ is time averaged quantities, $k+1/2$ is time edge quantities.
These equations contain a spatial---the angular flux at the cell midpoint is a simple average of the two half-cell average quantities---and a time---\textit{upstream} prescription for the cell-edge angular flux---closures.
This defines quantities of interest on the stencil shown in figure \ref{fig:stencil}.

\begin{figure}[!htb]
    \centering
    \includegraphics[width=\textwidth]{figures/stencil.png}
    \caption{The discretization stencil}
    \label{fig:stencil}
\end{figure}

%Figure \ref{fig:stencil} shows the stencil location for angular flux and source terms. 

In the traditional source iteration, the scattering source is presumed known from a previous iteration, which leads to the following set of equations to be solved in transport ``sweeps''.
This means that new estimates of both the end of time-step value of angular flux and time-averaged angular flux are computed together in each cell ordinate and group. 
So a $4\times4$ sized system (from stencil values) of equations must be solved to go from cell to cell in each angle.
This process is annoyingly serial over space, but embarrassingly parallel over angles.
The whole system sparse matrix is block (in each ordinate) lower triangular (over all cells).
Source iterations couple angles together in the iterative method by use of the scalar flux.

In OCI, the scattering source is subtracted to the left-hand side, and the information that comes from cells other than cell $j$ is assumed to be known from a previous iteration.
This means that all $4NG$ angular fluxes ($N$ angles, $G$ groups, and stencil values $L$, $R$, $k$, $k+1/2$) are computed simultaneously in cell $j$.
For simple corner balance in slab geometry, this means there is a local $4NG \times 4NG$ system of equations to be solved in each cell.


\subsection{Fourier Analysis}

To ensure that our combination of higher-order discretization schemes is still unconditionally stable, we performed a Fourier analysis to find the slowest-converging mode of the method.
Assuming no scattering, no sources, homogeneous, mono-energetic problem, and make physical assumptions (only positive values of $\Delta x$, $\Delta t$, $\Sigma$, and $v$) we derived the space-time discritization scheme's eigen-function and numerically solve it (see Appendix~\ref{app:therefore}).
% \begin{subequations}
% \begin{align}
%     \psi_{k+1/2,j,L} &= \theta^{k+1}a e^{i\omega j}
%     &
%     \psi_{k+1/2,j,R} &= \theta^{k+1}b e^{i\omega j}
% \end{align}
% \begin{align}
%     \psi_{k,j,L} &= \theta^{k}c e^{i\omega j}
%     &
%     \psi_{k,j,R} &= \theta^{k}d e^{i\omega j}
% \end{align}
% \begin{align}
%     \psi_{k-1/2,j,L} &= \theta^{k}a e^{i\omega j}
%     &
%     \psi_{k-1/2,j,R} &= \theta^{k}b e^{i\omega j}
% \end{align}
% \end{subequations}
% where $k$ is the mode number, $i$ is the imaginary number, $\omega$ is the frequency of the mode, $\theta$ is the , and $j$ is the specific cell.
% Substituting our ansätze into Eqs.~\eqref{eq:scb-mb-a}, \eqref{eq:scb-mb-b}, \eqref{eq:scb-mb-c}, and \eqref{eq:scb-mb-d}, respectively, we
% When this system is solved for its eigenvalues ($\Lambda$) numerically it shows that a MB-SCB scheme is unconditionally stable over $\mu \in [-1, 1]$.  

\subsection{Verification using method of manufactured solution}

Finding benchmark problems that simulate transient, energy dependent, single dimension solutions to the neutron transport equation is difficult \cite{roy_review_2005}.
I used the method of manufactured solutions (MMS) to derive my own benchmark \cite{nse_mms_warsaw}.
It is widely accepted in the field and has previously been used to verify other deterministic neutron transport codes including the production code solver MOOSE \citep{wang_application_2018,moosemms}.

MMS follows a backward solving technique where a solution is made-up---expressing any desired behavior (e.g., smoothness, differentiability, positivity)---then inserted into the governing equation and solved for the source term.
The manufactured source is supplied to the solver to verify that it converges to the correct solution. %at the expected rate. 
To implement MMS I start with a manufactured continuous functional representation of the angular flux in angle, space and time adapted from Eq.~(7) in reference \cite{wang_application_2018} for time dependence.
When moving to multi-group two equations will be used to define the flux in group one and two with the group to group scattering operator linking them together.
%\begin{subequations}
%    \label{eq:manufactured_sol}  
%    \begin{align}
%        \psi_1(x,t,\mu) = (1-\mu^2)\sin(\pi x)e^{-t} \label{eq:af_mms_cont1}\\
%        \psi_2(x,t,\mu) = (1-\mu^2)(-x^4 + x + 1)e^{-t/2} \label{eq:af_mms_cont2}
%    \end{align}
%\end{subequations} 

These equations can now be inserted into the NTE (Eq.~\eqref{eq:sn_nte}) and solved for a continuous functional representation of the source $Q$.
From there the source equations are placed on the stencil by averaging (via numerical integration) in space and time where necessary (see figure \ref{fig:stencil}).
This discretized set of sources is then supplied to the solver and the converged angular flux it provides is then compared to the original manufactured source (again placed on the stencil).
%The range we will be solving these equations is $x \in [0,1]$, $t \in [0,5]$ and $\mu \in [-1,1]$.
%Now we take \ref{eq:af_mms_cont1} and \ref{eq:af_mms_cont2} and insert them into the $S_N$ slab wall neutron transport equation defined in Eqn. \ref{eq:sn_nte} 
%\begin{subequations}
%        \begin{multline}
%            \frac{1}{v_1} \frac{\partial \psi_1(x,t,\mu)}{\partial t} + \mu \frac{\partial \psi_1 (x,t,\mu)}{\partial x} + \Sigma_1\psi_1(x,t,\mu)\\ 
%            =\frac{1}{2} \left(  \Sigma_{s,1} \int^1_{-1}\psi_1(x,t,\mu) \partial \mu + \Sigma_{s,2\rightarrow 1} \int^1_{-1}\psi_2(x,t,\mu) \partial \mu + Q_1\right) 
%        \end{multline}
%        and
%        \begin{multline}
%            \frac{1}{v_2} \frac{\partial \psi_2(x,t,\mu)}{\partial t} + \mu \frac{\partial \psi_2 (x,t,\mu)}{\partial x} + \Sigma_2\psi_2(x,t,\mu)\\ 
%            =\frac{1}{2} \left(  \Sigma_{s,2} \int^1_{-1}\psi_2(x,t,\mu) \partial \mu + \Sigma_{s,1\rightarrow 2} \int^1_{-1}\psi_1(x,t,\mu) \partial \mu + Q_2\right) 
%        \end{multline}
%\end{subequations}
%which yields continuous functional representations for solutions a source term,
%\begin{subequations}
%        \begin{equation}
%            Q_1(x,t,\mu) = \frac{2}{v_1} + 2\mu + 2\Sigma_1(\mu + t + x) - \Sigma_{S,1}(2t + 2x) - 2\Sigma_{S,2\rightarrow1}tx^2 
%        \end{equation}
%        and
%        \begin{equation}
%            Q_2(x,t,\mu) = \frac{2x^2}{v_2} + 4\mu tx + 2\Sigma_2(\mu + tx^2) + \Sigma_{S,1\rightarrow2}(2t + 2x) - 2\sigma_{S,2}tx^2
%        \end{equation}
%\end{subequations}
%Note that these equations are fully defined by independent variables and material data.


\subsection{Implementation in code}

I use the ROCm compute library to solve the system of equations.
Modern GPU vendor-supplied LAPACK libraries often include a \texttt{strided\_batched} class of solvers.
These will operate on a group of like-sized systems in unison and are optimized by the vendors of the hardware themselves.
For example, LU decomposition with pivoting (the most generic direct solver for a system of linear equations) is implemented using rocSOLVER \texttt{strided\_batched\_dgsev} \cite{rocsolver}.
Direct solvers are used in this work.
Currently all systems are assumed dense which for OCI might not be prudent.
Future work could look into using sparse direct solvers which are similarly available as their dense siblings.
%Spare direct solvers could be utilized in the future to decrease memory footprint of OCI schemes specifically.
This makes the use of a strided batched implementation of LU decomposition with pivoting ideal.
Furthermore in this work once that system is solved the return of the A matrix is automatically the decomposition: $L+U+D$.
In subsequent iterations this system can be back solved very quickly again using library functionality (RQ1).
For both schemes the entire convergence loop was executed on the GPU (see appendix \ref{app:therefore}).
Profiling was used to ensure that both sets of compute kernels where performant.

\subsection{Exploration of acceleration scheme}
\label{ssec:method_acc}

This section discusses ongoing work: the derivation of an acceleration scheme for OCI in the thin limit.
When working with iterative schemes, it becomes useful to define systems in operator notation.
The NTE can be written generally as
\begin{equation}
    H\psi = q
\end{equation}
where $H$ is the transport operator, $\psi$ is the unknown function for angular flux and $q$ is some source.
To solve this equation H can be split into components that make sense given the NTE, then lagging and leading components can be established (assuming a fixed point iteration).
How the operator is split is what differentiates OCI from SI.
For example if $H = L-S$ where $L$ is the left hand side of the transport equation and $S$ is the scattering operator and lag the angular fluxes on the scattering operator,
\begin{equation}
    L\psi^{l+1} = S\psi^l + q
\end{equation}
which is a source iteration where $L$ = streaming thru space and time and total absorption operator, %\\ ($\mu_m\frac{\partial\psi}{\partial x} + \frac{1}{v}\frac{\partial\psi}{\partial t} + \Sigma \psi$)
$S$ is the scattering operator (discrete-to-moment operator) which couples all the simultaneous PDEs in angle and group together, %\\($\Sigma_s \int^1_{-1}\psi\partial\mu = \Sigma_s \sum_{n=0}^{M}w_n\psi_n = \Sigma_s\phi$)
$\psi$ angular flux (dependent variable of interest), and
$q$ is the fixed material source.
This is where Rosa, Warsaw, and Perks (2013)~\cite{rosa_cellwise_2013} start in there anylisys of OCI which they call cellwise-bJ.
OCI works by lagging the incident cell wall fluxes of the neighboring cells.
Those fluxes come from the upstream closure that transport solvers use.
Thus it is convenient to further split L,
\begin{equation}
    L = L_c + L_b
\end{equation}
where $L_c$ are the angular fluxes within a mesh cell and $L_b$ are the angular fluxes incident to the mesh cell's surface (boundary).
    %\\($\mu_m\psi_b\frac{dA}{dx} + \frac{1}{v}\psi_b\frac{dA}{dx}$)
Note that what $L_b$ actually consists of depends on the spatial discretization.
After this operator splitting Eq.~\eqref{eq:operator_nte} now becomes
\begin{equation}
    \psi[I+(L_c-S)^{-1}L_b] = (L_c-S)^{-1}q
\end{equation}
where $I$ is the identity matrix.
When OCI's prescribe of what lags (incident fluxes on the surface of the cell) is used this becomes
\begin{equation}
    \label{eq:itter}
    \psi^{(l+1)} = (L_c-S)^{-1}(-L_b\psi^l+q)
\end{equation}

When looking for an acceleration scheme examining the structure of $L_b$ becomes important.
In my research I use the TDMB+SCB discretization for time dependent single dimension multi-group Sn problems (see Eq.~\eqref{eq:tdmb+scb}).
If the angular fluxes from the boundary of the cell are identified (denoted by $j\pm1/2$) $L_b$'s structure can be generally identified.
It should be something like $\pm\mu_m$ filled along four diagonals with every 4$^{th}$ space filled.
This is also vaguely similar to how the SI equations look if formed as a lower triangular system.
In higher dimensions I expect the structure to remain mostly the same for rectilinear grids but might get more complex for unstructured meshes where surface areas may become an issue.

Next I examine the within cell operator: $L_c$.
$L_c$ contains the differential and absorption terms from the NTE.
Also, OCI's converge rate decays in the thin limit due to a-synchronicity \cite{rosa_cellwise_2013, hoagland_hybrid_2021}, which is indicated by a small mean-free path (MFP).
So a small MFP leads to a large condition number of our iteration matrix, thus slow convergence.
To isolate this term I use operator splitting again
\begin{equation}
    L_c = K_d + K_a
\end{equation}
where $K_d$ is the differential operators and
\begin{equation}
    K_a = \frac{\Delta x \Sigma}{2}  I
\end{equation}
is the absorption term.
$K_a$'s structure is simple and clear from the equations \ref{eq:tdmb+scb}.
I could pull out a solitary $\Sigma$ but in this form I can identify that
\begin{equation}
    K_a = \frac{\text{MFP}}{2}  I
\end{equation}
Pushing forward I break up our equations into a standard synthetic approach.
I start with Eq.~\eqref{eq:itter} but recast it with some of the expansions identified,
\begin{equation}
    \psi^{l+1}(K_d + \delta I - S) = (-L_b\psi^l + q)
\end{equation}
where $\delta=$ MFP$/2$.
Now the associated equation for the iteration error is,
\begin{equation}
    (K_d + \delta I-S) f^{l+1} = -L_b f^l
\end{equation}
where $f^{l+1} = \psi^{\text{converged}}-\psi^{l+1/2}$ and $f^{l} = \psi^{\text{converged}}-\psi^{l}$.
Note that this equation is exact in it's current forum.
Now $-L_bf^{l-1}$ is subtracted from both sides:
\begin{equation}
    (K_d + \delta I - S + L_b)f^{l+1} = L_b(f^{l+1}-f^l)
\end{equation}
Notably if everything re-combined Eq.~\eqref{eq:itter} is reformed.
So a synthetic acceleration system starting with Eq.~\eqref{eq:itter} rewritten for half steps is
\begin{subequations}
    \begin{equation}
        \label{eq:acc1}
        \psi^{(l+1/2)} = (L_c-S)^{-1}(-L_b\psi^l+q)
    \end{equation}  
    the associated equation for mid step error
    \begin{equation}  
        \label{eq:acc2}
        f^{l+1/2} = -M L_b(\psi^{l+1/2}-\psi^l)
    \end{equation}  
    and the definition of the errors
    \begin{equation}  
        \label{eq:acc3}
        \psi^{l+1} = \psi^{l+1/2} + f^{l+1/2}
    \end{equation}
\end{subequations}
where,
\begin{equation}
    M \approx (K_d + \delta I + L_b - S)^{-1}
\end{equation}
and is easier to obtain then the actual inverse.
Note that equation 3 from ref \cite{tsa2009rosa} is the recombined version of Eq.~\eqref{eq:acc2}.
From here estimations can be made about what a good approximate of $M$ will be in the conditions where Eq.~\eqref{eq:acc1} degrades.
%If we actually inverted $H$ this would just be another transport step.
%Note that we can relate this idea of synthetic acceleration to preconditioning by noting
%\begin{equation}
%    P \equiv I + M(I-H)
%\end{equation}
%From here I take some stabs at identifying what reasonable values for $M$ could be.
%When we ask our selves "what should M be" we could just as easily pose the more pessimistic analog: "why or, where rather, does H suck to invert".
%This is where our knowledge about the thin limit comes in.
%We know H is painful to invert when $\delta I$ is small as that is a problem in the thin limit where cell wise communication becomes dominant.
%When $\delta I$ is small a good conservative approximation for M may become
%\begin{equation}
%    H^{-1} \approx M = (K_d + L_b - S)^{-1}
%\end{equation}
%but we just converged our within-cell scattering, time rate of change, and steaming operators from the transport step so lets assume we have a good estimate of that.
%We can also assume that the boundary information coming into every cell will blah likewise lets scoot that and all we are left with is the differential operators (from a discretization scheme)
%\begin{equation}
%    H^{-1} \approx M = (K_d)^{-1}
%\end{equation}
%which makes the full mid step solution
%\begin{equation}
%    \label{eq:m}
%    -K_d^{-1}*L_b
%\end{equation}
%This leads to the suggestion at the possible algorithm
%\begin{enumerate}
%    \item transport step as described in eq. \ref{eq:acc1}
%    \item mid step corrector (eq. \ref{eq:acc2}) with . This will look like a transport in every angle
%    \item colsure from eq
%\end{enumerate}
%But Joanna, your probably asking yourself. 
%This just using a cheap sweep between every OCI transport iteration
%I say in 1D
%in higher dimensions this will look like a \textit{a ray trace} something GPUs are literally built for.

%From here more work is need to be down to conduct Fourier analysis in both with and with out the discretization scheme

%What ever scheme is used it can be supplemented by GMRES

\section{Monte Carlo transport methods}
As in the previous section the methods employed are further described in attached publications in the appendix.
They are briefly summarized here.

%   
\subsection{Python-based portability for rapid methods development}

A software engineering scheme was developed and deployed in Monte Carlo/Dynamic Code \cite{morgan_monte_2024} using the Numba compiler for Python \cite{lam_numba_2015}, mpi4py \cite{mpi4py_2021}, and an a-synchronous GPU event scheduler called Harmonize \cite{brax2023} (See Appendix~\ref{app:cise}).
MC/DC can run the same set of physics functions written by subject-area experts in the Python interpreter or compiled to x86-64, ARM64, and Power9 CPUs as well as AMD and Nvidia GPUs.
All compilation targets also support the use of MPI to distribute work to multiple nodes of an HPC.
It is a demonstration of portability using a high-level language coupled to a portability framework to enable rapid methods development.

\subsection{Delta tracking in MCATK and MC/DC}

In MCATK a hybrid delta-surface tracking scheme was developed and deployed (See Appendix~\ref{app:hybridmcatk}).
I initially implemented this scheme (my work is fully described by the publication included in Appendix \ref{app:hybridmcatk}), but it has been extended by others and is shipped in current production versions of MCATK.
The point of the delta-surface tracking scheme in MCATK is entirely to avoid extra expensive macroscopic cross section lookups.
It still conducts surface tracking on mesh.
After implementation the hybrid delta-tracking scheme was tested in both k-eigenvalue and fixed source transient simulations using the
\begin{itemize}
    \item HEU-MET-FAST-086 benchmark of Godiva IV \cite{godivaiv2021}, and
    \item Measurement of Uranium Subcritical and Critical IER 488 (MUSiC) experiment \cite{music2021}.
\end{itemize}
Runtime and errors were then compared to draw conclusions.
This work was published in a conference publication \cite{morgan2023oci} (see Appendix~\ref{app:hybridmcatk}).

In MC/DC I propose a slight alteration to the hybrid-scheme developed and deployed in my previous work in MCATK.
Current versions of MC/DC do not stop particles at mesh crossings when using surface tracking and use a track-length estimator to tally impacts to multiple cells at once.
Using this tally algorithm already in MC/DC, delta tracking could be used in conjunction with the track length estimator as in the previous work in MCATK.
However the proposed algorithm in MC/DC will more closely resemble the vanilla delta tracking algorithm and not stop particles at now cell \textit{and} surface crossings.
Other quantities of interest will be tallied with collision estimator as the specific material data is required for values like fission reaction rate.
This will require additional computational burden but the hope is that the new tally functionality in MC/DC is sufficiently optimized to still allow this scheme to be more performant.

To evaluate the delta tracking scheme as compared to surface tracking I will use figure of merit (FOM) values.
FOMs are constructed to observe impact of a variance reduction technique on both variance (as computed by the Monte Carlo scheme) and on runtime (as many variance reduction schemes incur additional computation burden).
The most common figure of merit and the one I will employ is simply
\begin{equation}
    FOM = \frac{1}{\sigma^2 t_{wc}}
\end{equation}
where $\sigma^2$ is the variance as already computed by MC solvers and $t_{wc}$ is the total wall-clock-runtime.
Using this metric (higher is better) variance reduction shames can be compared to each other and the analog solver.
I propose using this metric to evaluate the impact of the hybrid-delta tracking scheme 
I plan to initially verify the tracking algorithm in MC/DC using the AZURV1 benchmark problem \cite{ganapol_homogeneous_2001}.
Then compare figures of merrit on problems of interest.
Potential problems of interest include:
\begin{enumerate}
    \item Dragon burst problem \cite{kimpland2021dragon};
    \item C5G7 Small modular reactor \cite{jia_hou_oecdnea_2017}; and
    \item Kobyashi void dog-leg void problem (multigroup and contentious energy simulations) \cite{Kobayashi2001}.
\end{enumerate}