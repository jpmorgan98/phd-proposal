\chapter{Objectives}
\label{ch:objectives}

\section{GPU enabled time dependent neutron radiation transport}
This exploration attempts to answer RQ 1.
\begin{enumerate}
    \item \textbf{DONE} Compare portability frameworks for Monte Carlo neutron transport deployment on both GPU and CPU platforms in a transient code
    \item \textbf{DONE} Support the production of a full Monte Carlo transport code and aid in the deployment to other accelerators
    \item \textbf{Submitted} Publish this work for a general software engineering audience
    \item \textbf{ONGOING} Conduct GPU performance analysis for fully time dependent problems of interest and publish
\end{enumerate}


\section{Explorations into GPU optimized deterministic schemes}
This investigation attempts to answer RQ 1 and 2, 3.
\begin{enumerate}
    \item \textbf{DONE} Derive discretization scheme for time dependent, multi-group, Sn, neutron transport in slab geometry
    \item \textbf{DONE} Conduct Fourier analysis to show the stability of the discretization scheme;
    \item \textbf{DONE} Implement a mono-energetic proof of concept of discretization schemes in both Source iterations and One-cell inversions in a Python code for both CPUs and GPUs;
    \item \textbf{DONE} Extend simulation to be energy dependent with the multi-group assumption and explore in C++;
    \item \textbf{ONGOING} Use the method of manufactured solution to verify the discretization scheme in problems and measure its convergence rate;
    \item \textbf{DONE} Implement discretization scheme in both a Source iteration and One-cell inversion iterative scheme on GPUs using vendor supplied linear algebra libraries to abstract device kernels, and measure wall clock runtime; 
    \item \textbf{DONE} Investigate transient effects on convergence rate of one-cell inversion and compare to source iteration; and
    \item \textbf{ONGOING} Publish this work.
\end{enumerate}

\section{Exploration of acceleration schemes for One-cell inversion}
This exploration attempt to answer RQ 4
\begin{enumerate}
    %\item \textbf{ONGOING} Implement diffusion synthetic acceleration to Source iteration to have a comparison. This will eventually serve as a comparison;
    \item \textbf{ONGOING} Derive and postulate various potential acceleration schemes, then down select to one or two schemes
    \item Do fourier analysis on selected scheme
    \item Implement acceleration scheme in code
    \item Verify convergence to correct value using method of manufactured solutions and/or Monte Carlo results
    \item Explore convergence rate in various problems
    \item Publish this work
\end{enumerate}

\section{Exploration of delta tracking using a hybrid tally}
This study attempts to answer RQ 5
\begin{enumerate}
    \item \textbf{DONE} Implement the hybrid-delta tracking algorithm on a structured mesh in MCATK
    \item \textbf{DONE} Compute runtime for speedup in materially complex problems using this novel scheme
    \item \textbf{DONE} Publish this work
    \item \textbf{ONGOING} Compute the macroscopic majorant cross section for an entire problem as a pre-process in MC/DC;
    \item Under a boolean condition---set by the user at runtime---add rejection sampling and distance to collision sampling using the majorant;
    \item When running in delta tracking mode use a path length estimator to tally the scalar flux and a collision estimator to tally other quantities of interest (i.e. reaction rates);
    \item Test using a problem of interest and compare FOM for quantities of interest;
    \item Publish this work
\end{enumerate}

\section{Possible difficulties and contingencies}
The largest and most risky open question left in this research is weather we can find an acceleration scheme for one cell inversions in the thin limit.
While explorations that result in a negative result are still essential and worth while, I list a few contingencies if acceleration schemes prove more difficult to come by:
\begin{enumerate}
    \item Investigations into a multi-grid multi-group scheme enabled by the solution of all angles and groups within a cell
    \item 2D explorations of OCI with further performance analysis on GPUs using roofline models
    \item Focusing an additional investigations into delta tracking in MC/DC, specifically looking at an adaptive energy scheme, where high energies with large MFPs use delta tracking and low energies use surface tracking
\end{enumerate}

