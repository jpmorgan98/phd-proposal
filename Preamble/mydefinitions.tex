%----------------------------------------------------------------------------------------
%	COMMANDS FOR THE THESIS
%----------------------------------------------------------------------------------------

\newcommand{\name}{Joanna Piper Morgan} % Author name
\newcommand{\thesistitle}{Piper's Ponderings} % Title of the thesis
\newcommand{\submissiondate}{September, 2024} % Submission date "Month, year"
\newcommand{\advisor}{Kyle E. Niemeyer} % Supervisor name
%\newcommand{\coadvisor}{Someone Someone} % Co-Supervisor name, comment this line if there is none


%----------------------------------------------------------------------------------------
%	BIBLIOGRAPHY STYLE (pick the style you want)
%----------------------------------------------------------------------------------------

\usepackage[square, numbers, sort&compress]{natbib} % for bibliography - Square brackets, citing references with numbers, citations sorted by appearance in the text and compressed (as in [4-7])
%\usepackage[longnamesfirst,round]{natbib} % Natural Sciences bibliography

\bibliographystyle{Preamble/physics_bibstyle} % You may use a different style adapted to your field
%\bibliographystyle{unsrtnat} % You may use a different style adapted to your field


%----------------------------------------------------------------------------------------
%	PACKAGES (be careful of package interaction)
%----------------------------------------------------------------------------------------

\usepackage{amsmath, amsthm, amssymb, amsfonts, amstext, bm, bbm, mathtools} % Math symbols
\usepackage{graphicx, subfigure, caption, epsfig, psfrag} % Figures
\usepackage{threeparttable, colortbl, multirow} % Tables
\usepackage{soul, pifont, color, hyperref, bbding, textcomp, manfnt, latexsym, wasysym} % Fonts
\usepackage{enumitem}
\usepackage[ruled, vlined]{algorithm2e}
% \usepackage[makeroom]{cancel}
\usepackage{lscape}
\usepackage{pdfpages}

%----------------------------------------------------------------------------------------
%	DEFINITIONS AND COMMANDS
%----------------------------------------------------------------------------------------

% Defining a theorem box for Criteria
\newtheorem{critere}{Criterion}
\newcommand{\crit}[2]{
\begin{center}  
\fbox{ \begin{minipage}[c]{0.9 \textwidth}
\begin{critere}
\textbf{\textup{ #1}} --- #2
\end{critere}
\end{minipage}  } \end{center}
}

\newcommand{\todo}[1]{\textcolor{red}{(#1)}}
% todo command from https://tex.stackexchange.com/questions/247681/how-to-create-checkbox-todo-list
% \usepackage{enumitem}
\newlist{todolist}{itemize}{2}
\setlist[todolist]{label=$\square$}
% \usepackage{pifont}
\newcommand{\cmark}{\ding{51}}
\newcommand{\xmark}{\ding{55}}
\newcommand{\done}{\rlap{$\square$}{\raisebox{2pt}%
{\large\hspace{1pt}\cmark}}\hspace{-2.5pt}}
\newcommand{\cancel}{\rlap{$\square$}{\large\hspace{1pt}\xmark}}

\newcommand{\eg}{\textit{e.g.}}
\newcommand{\ie}{\textit{i.e.}}

% General
\newcommand{\bea}{\begin{eqnarray}} % Shortcut for equation arrays
\newcommand{\eea}{\end{eqnarray}}
\newcommand{\defin}{\triangleq}
\newcommand{\pdiff}[2]{ \frac{ \partial #1 }{ \partial #2 } }
\newcommand{\red}[1]{{\color{red}\textbf{#1}}}
\newcommand{\blue}[1]{{\color{blue} #1}}
\newcommand{\green}[1]{{\color{ForestGreen}\textbf{#1}}}
\newcommand{\tal}[1]{\textbf{\alert{#1}}}

% Attenuation equation
\newcommand{\StmD}{\Sigma_t^\Delta \Delta x}
\newcommand{\StdD}{\Sigma_t^\Delta \Delta x}
\newcommand{\Stm}{\Sigma_{t,m}^{0}}
\newcommand{\Std}{\Sigma_{t,m}^{\Delta}}

% Var and EE
\newcommand{\Var}[1]{\mathbb{V}ar\left[#1\right]}
\newcommand{\EE}[1]{\mathbb{E}\left[#1\right]}
\newcommand{\Cov}[2]{\mathbb{C}ov\left[#1,#2\right]}

% Xi
\newcommand{\EExi}[1]{\mathbb{E}_\xi\left[#1\right]}
\newcommand{\Vxi}[1]{\mathbb{V}ar_\xi\left[#1\right]}
\newcommand{\Nxi}{{N_\xi}}
\newcommand{\xii}{\xi^{(i)}}
\newcommand{\sumxi}{\sum_{i=1}^{\Nxi}}

% Eta
\newcommand{\EEeta}[1]{\mathbb{E}_\eta\left[#1\right]}
\newcommand{\Veta}[1]{\mathbb{V}ar_\eta\left[#1\right]}
\newcommand{\Neta}{{N_\eta}}
\newcommand{\etaj}{\eta^{(j)}}
\newcommand{\etak}{\eta^{(k)}}
\newcommand{\Zeta}{ {Z_\eta} }
\newcommand{\muRT}{\hat{\mu}_{\SigSqRT}}
\newcommand{\sumeta}{\sum_{j=1}^\Neta}

% Estimators
\newcommand{\Qpoll}{\tilde{Q}}
\newcommand{\Qhat}{\hat{Q}}
\newcommand{\SigSqeta}{\sigma^2_\eta}
\newcommand{\SigSqRT}{\sigma^2_{RT,\Neta}}
\newcommand{\hatSigSqeta}{\hat{\sigma}^2_\eta}
\newcommand{\fij}{f(\xii,\etaj)}
\newcommand{\f}{f(\xi,\eta)}
\newcommand{\ft}{\tilde{f}}

% Omega
\newcommand{\EEom}[1]{\mathbb{E}_\omega\left[#1\right]}
\newcommand{\Vom}[1]{\mathbb{V}ar_\omega\left[#1\right]}
\newcommand{\Nom}{{N_\omega}}
\newcommand{\EExo}[1]{\mathbb{E}_{\xi,\omega}\left[#1\right]}
\newcommand{\Vxo}[1]{\mathbb{V}ar_{\xi,\omega}\left[#1\right]}
\newcommand{\Qxo}{Q\left(\xi,\omega\right)}
\newcommand{\PE}{ {\mathbb{P}_{\mathbb{E}}} }
\newcommand{\Pxi}{ \mathbb{P}_{\mathbb{E | \xi}} }
\newcommand{\PEpollom}{ \tilde{\mathbb{P}}^{\mathbb{E}}_{\Nom} }

% SM Example problem
\newcommand{\T}[1][]{T^{#1}\left( \xi, \omega \right)}
\newcommand{\PA}{p_1}
\newcommand{\SA}{\Sigma_{t,1}}
\newcommand{\SB}{\Sigma_{t,2}}
\newcommand{\SC}{\Sigma_{t,3}}
\newcommand{\StC}{\Sigma_{t,3}^{0}}
\newcommand{\StCd}{\Sigma_{t,3}^{\Delta}}
\newcommand{\Dx}{ \Delta x}
\newcommand{\DxA}{ \Delta x_A}
\newcommand{\DxB}{ \Delta x_B}
\newcommand{\Nt}{N_{tot}}
\newcommand{\NA}{N_1(\omega)}
\newcommand{\FAB}{F(\xi_A,\xi_B)}
\newcommand{\Fm}{F_0}
\newcommand{\Fd}{F_{\Delta}(\xi_A,\xi_B)}
\newcommand{\xiA}{\xi_A}
\newcommand{\xiB}{\xi_B}
\newcommand{\Bom}{B_\omega(x)}
\newcommand{\Bomy}{B_\omega(y)}
\newcommand{\intsin}[1]{\frac{ \sinh\left[ #1 \right] }{ #1 }}
\newcommand{\e}[1]{\exp \left( #1 \right) }
\newcommand{\sinhh}[1]{\text{sinh}\left[ #1 \right]}
\newcommand{\g}{\, | \,}


% PCE
\newcommand{\xiu}{{\xi_u}}
\newcommand{\xinu}{{\xi_{\sim u}}}
\newcommand{\Nsi}{{N_{SI}}}
\newcommand{\EEu}[1]{\mathbb{E}_{\xiu} \left[#1\right]}
\newcommand{\EEnu}[1]{\mathbb{E}_{\xinu} \left[#1\right]}
\newcommand{\Vu}[1]{\mathbb{V}ar_{\xiu}\left[#1\right]}
\newcommand{\Vnu}[1]{\mathbb{V}ar_{\xinu} \left[#1\right]}
\newcommand{\Ppoll}{\tilde{P}_\Nsi}
\newcommand{\hf}{\mathrm{HF}}
\newcommand{\LF}{\mathrm{LF}}
\newcommand{\kn}{{\Large\dbend\normalsize\hspace{1mm}}}
\newcommand{\PCE}{Q^{PC}}
\newcommand{\PCvar}{\hat{\sigma}^{2}_{PC}}

